%preface.tex
\section*{前言}
\label{sect:preface}

本文档是 ``Using Imported Graphics in \LaTeX\ and \pdfLaTeX'' 的中文译本,并加入少许译者自己的使用心得所成。

在用 \LaTeX\ 编写论文、书稿时,经常要遇到使用图形的情况。
本书将讲述如何在 \LaTeX\ 文件中插入图片以及与此有关的一些其它问题。
如果您想查找某一特定的信息,那么可以查阅第~\pageref{toc}~页的\hyperref[toc]{目录}和第~\pageref{sec:index}~页的\hyperref[sec:index]{索引}。

插图的基本步骤是,首先确认宏包的导入
\begin{Verbatim}
\usepackage{graphicx}
\end{Verbatim}
然后使用 \verb|\includegraphics| 命令来插图
\begin{Verbatim}
\includegraphics{file}
\end{Verbatim}
该命令的更多细节将在第~\ref{sect:}~节介绍(\pageref{sect:})。

本文档分为如下的五个部分。

\paragraph{第一部分:背景介绍}
在这一部分中,介绍了一些有关的历史资料和基本的 \LaTeX{} 术语。
此外也包括
\begin{itemize}
	\item Encapsulated PostScript (eps) 图形格式,eps 与 ps 文件的区别,
	以及将其它图形格式转为~EPS~格式的方法。
	\item 通过 \pdfTeX 引擎可以直接导入的图片格式(jpeg, png, pdf, MetaPost)。
	\item 一些与图形有关的自由软件和共享软件。
\end{itemize}

\paragraph{第二部分:\LaTeX{} 图形宏包套件}
在这一部分中,详细介绍了 \LaTeXe{} 图形宏包套件中用于引入、缩放和旋转图形的命令。
这部分涵盖了~\LaTeXe{}~图形宏包的文档的大部分内容\cite{grfguide}。

\paragraph{第三部分:\LaTeXe{}~图形命令的使用}
这一部分介绍了如何使用~\LaTeXe{}~图形宏包套件中的命令来引入、缩放和旋转图形。
此外还讨论了以下三种情况:
\begin{itemize}
	\item 在支持管道(例如 UNIX )的系统中,使用 \texttt{dvips} 还可以实时地插入压缩的 EPS 图形或其它格式的图形(\prgname{tiff}, \prgname{gif}, \prgname{jpeg}, \prgname{pict} 等)。
	在其它的系统中,非 \prgname{eps} 格式图形必须先转换为 \prgname{eps} 才行。
	因为 \LaTeX\ 还是 \texttt{dvips} 都没有内置解压缩和转换图像格式转换的功能,
	所以相应软件需要使用者提供。
	\item 由于许多图形应用程序只支持 ascii 文本格式,
	\pkg{psfrag} 宏包可以将 \prgname{eps} 图形中的文字替换为 \LaTeX{} 符号或数学表达式。
	\item 当一个\prgname{eps}图形被多次使用时(比如文字后面或页眉上的徽标),最后生成的 \prgname{ps} 文件会包含此\prgname{eps}图形的很多个副本。
	当所使用的图形不是位图格式时,可以通过定义 \prgname{PostScript} 命令来避免此图形被重复插入,
	从而减小得到的\prgname{eps}文件体积。
\end{itemize}

\paragraph{第四部分:\env{figure} 环境}
将插入的图形放置于\env{figure} 环境中有几个优点。
在\env{figure}环境中的图形会被自动编号,从而可被引用或加到目录中。
因为置于该环境中的图形可以通过浮动来很好分页,所以可以很容易的制作出具有专业水准的文稿。
除了介绍\env{figure}环境的一般信息外,这一部分还讲述了如下一些和图形有关的话题:
\begin{itemize}
	\item 怎样自定义\env{figure}环境,例如怎样调整图形的放置位置、图形周围的距离、
	标题的距离,以及在图形与文本之间加入分隔线等。
	还可以自定义标题的格式,改变标题的样式、宽度和字体等。
	\item 怎样创建边注图形和伸入边注位置的宽图形。
	\item 怎样在纵向页面版式的文档中加入横向图形。
	\item 怎样将标题放置于图形的两边而不是上下。
	\item 对于双面排版的文档,怎样确保一幅图形放置于奇数页或偶数页。
	还有,怎样确保两幅图形都出现在同一对开页面上。
	\item 怎样得到带框的图形。
\end{itemize}

\paragraph{第五部分:复杂图形}
这一部分介绍如何创建较为复杂的图形,以同时包含多个图像。
\begin{itemize}
	\item 怎样形成并列的图形、图像和子图。
	\item 怎样在一个浮动体内将一个表格放置于图形之后。
	\item 怎样堆叠多行图形。
	\item 怎样创建跨页的连续图形。
\end{itemize}

\subsection*{如何得到本文档}
本书的\prgname{ps}格式和\prgname{pdf}格式的英文原版网络位置为
\begin{center}
	\href{ftp://ctan.tug.org/tex-archive/info/epslatex/english/epslatex.ps}{CTAN/info/epslatex/english/epslatex.ps}\\
	\href{ftp://ctan.tug.org/tex-archive/info/epslatex/english/epslatex.pdf}{CTAN/info/epslatex/english/epslatex.pdf}
\end{center}
其中 CTAN (Comprehensive \TeX{} Archive Network) 可以用如下站点或其映像站点中代替:
\begin{center}
	\begin{tabular}{ll}
	England & \url{ftp://ftp.tex.ac.uk/tex-archive/} \\
	Germany & \url{ftp://ftp.dante.de/tex-archive/} \\
	Denmark & \url{ftp://tug.org/tex-archive} \\
	France & \url{ftp://ftp.loria.fr/pub/ctan} \\
	Russia & \url{ftp://ftp.chg.ru/pub/TeX/CTAN} \\
	Vermont, USA & \url{ftp://ctan.tug.org/tex-archive/} \\
	Florida, USA & \url{ftp://ftp.cise.ufl.edu/pub/mirrors/tex-archive/} \\
	Utah, USA & \url{ftp://ctan.math.utah.edu/tex-archive/} \\
	Korea & \url{ftp://ftp.ktug.or.kr/tex-archive/} \\
	Japan & \url{ftp://ftp.riken.go.jp/pub/tex-archive/} \\
	Hong Kong & \url{ftp://ftp.comp.hkbu.edu.hk/pub/TeX/CTAN/} \\
	Singapore & \url{ftp://ftp.nus.edu.sg/pub/docs/TeX/} \\
	New Zealand & \url{ftp://elena.aut.ac.nz/pub/CTAN} \\
	Australia & \url{ftp://ctan.unsw.edu.au/tex-archive/} \\
	India & \url{http://mirror.gnowledge.org/ctan/} \\
	South Africa & \url{ftp://ftp.sun.ac.za/CTAN/} \\
	Brazil & \url{ftp://ftp.das.ufsc.br/pub/ctan/}
	\end{tabular}
\end{center}
完整的 CTAN 镜像列表可以从任一CTAN站点的 CTAN.sites 获得。
本文档版本2.0的法文版本由Jean-Pierre Drucbert完成,
其 \prgname{pdf} 和 \prgname{PostScript} 文件可以从如下位置获得
\begin{center}
	\href{ftp://ctan.tug.org/tex-archive/info/epslatex/french/fepslatex.pdf}{CTAN/info/epslatex/french/fepslatex.pdf} \\
	\href{ftp://ctan.tug.org/tex-archive/info/epslatex/french/fepslatex.ps}{CTAN/info/epslatex/french/fepslatex.ps}
\end{center}

\subsection*{致谢}
略。

\subsection*{许可协议}
版权所有 \copyright{} 1995--2006  Keith Reckdahl.
可在 \LaTeX\ 项目公共协议(LPPL, \LaTeX\ Project Public License)x下复制和分发。

LPPL 协议的内容可参见 \url{http://www.latex-project.org/lppl/}。

%中文版的~PDF~
%格式文件和源文件可从~\href{ftp://ftp.ctex.org/pub/tex/documents/latex/graphics/}%
%{ftp.ctex.org}~下载。
%
%\vspace{1cm}
%\hbox to \textwidth{\hfill
%	\href{mailto:lwang2002@gmail.com}{%
%		\Large {\CJKfamily{kai}王~~磊}}}
%
%\hbox to \textwidth{\hfill{\CJKfamily{kai}二○○○年四月十三日}}

\clearpage
\endinput
